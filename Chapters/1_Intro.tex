\chapter{Introdução} 
\label{intro}

%----------------------------------------------------------------------------------------
%	Objetivos
%----------------------------------------------------------------------------------------

    \section{Objetivos}
    \label{goals}
    
        \paragraph{} A Variabilidade da Frequência Cardíaca (Heart Rate Variability - HRV) provê uma janela não-invasiva para funções do sistema nervoso autônomo diretamente associadas às reações de stress e relaxamento. Ela tem sido amplamente utilizada por esportistas para medição de capacidade física e recuperação de treinamento e pela comunidade de saúde como uma medida de resposta ao stress. Nosso principal objetivo nesse trabalho é explorar a informação contida nesse sinal e avaliar a viabilidade de seu uso em outras áreas, tais como seu potencial para ser utilizada em uma interface humano-máquina. 
        
        paragraph{} A principal motivação para isso é que o sinal de HRV é de fácil obtenção, sendo coletado através de sensores com eletrodos ou até mesmo relógios de pulso que meçam os intervalos entre batimentos cardíacos com precisão. Isso lhe permitiria ser uma alternativa viável, em algumas circunstâncias, ao uso de equipamentos mais caros e desconfortáveis para obtenção de informação via EEG.
        
        \subsection{Pipeline para tratamento de dados para classificação de HRV}
        
            \paragraph{} Para poder explorar a informação contida no sinal de HRV, nosso primeiro objetivo é definir e implementar um pipeline de dados que nos permita coletar o dado bruto de intervalos entre batimentos cardíacos, armazená-los em um banco de dados com anotações manuais que possam ser correlacionadas com eles e processá-los de forma a obter datasets completos que sirvam de entrada para classificadores. Desejamos que esse pipeline possa ser usado, primeiramente nos classificadores descritos a seguir, mas que seja flexível o suficiente para que possa servir de suporte a quaisquer pesquisas que desejem explorar o dado de HRV em conjunto com alguma outra variável.
        
        \subsection{Classificação supervisionada de atividades}
        
            \paragraph{} O primeiro tipo de informação que buscamos no dado de HRV processado é sobre as atividades diárias de um indivíduo. Como o tipo de atividade exercida pode influenciar a frequência cardíaca, desejamos responder se é possível, e com até que nível de granularidade é possível, classificar as atividades realizadas por um indivíduo durante seu dia-a-dia usando apenas a informação da frequência cardíaca. Essa classificação poderia servir como dado de apoio para fornecer contexto a sistemas de detecção mais detalhados via EEG, a aplicações de \textit{lifelogging}, ou até mesmo para análise em tempo real de respostas involuntárias ao estresse.
       
        \subsection{Classificação supervisionada de manipulação de HRV - IHC}
            
            \paragraph{} Uma vez entendidos quais parâmetros permitem a classificação de atividades, nosso próximo objetivo é tentar aumentar ou reduzir o HRV voluntariamente através de técnicas de \textit{biofeedback} para replicar controladamente as alterações nesses parâmetros. Com isso, esperamos conseguir controlar uma Interface Humano-Máquina (IHC), mesmo que de maneira simples. Por ser não-invasivo e obtido através de sensor razoavelmente confortável e discreto, essa interface poderia ter aplicações em áreas que exigem, hoje, \textit{hardware} mais sofisticado para essa interação.


%----------------------------------------------------------------------------------------
%	Revisão bibliográfica
%----------------------------------------------------------------------------------------

    \section{HRV}
    \label{HRV}
        
        \subsection {Definição e obtenção}
        \paragraph{} O coração saudável não tem um ritmo regular como o de um metrônomo. Seu ritmo sofre oscilações constantes e complexas, que permitem que o sistema cardiovascular se adapte rapidamente a alterações físicas e fisiológicas no ambiente ~\cite{Hansen2004HeartDetraining}. A Variabilidade da Frequência Cardíaca (HRV) é uma medida da variação dos intervalos entre batimentos consecutivos, chamados \textit{Interbeat Intervals} (IBI) ou intervalos RR. O nome de intervalo RR se deve ao fato de que esses intervalos são calculados como a distância entre dois picos do complexo QRS que é parte da forma de onda de um eletrocardiograma (ECG). ~\cite{TaskForceoftheEuropeanSocietyofCardiologytheNorthAmericanSocietyofPacing1996HeartUse}
        
        \paragraph{} A série de intervalos RR é derivada do próprio ECG, o qual mede os ciclos de polarização originados no nó sinoatrial, que fazem com que o átrio e o ventrículo se contraiam em ciclos, dando origem aos batimentos cardíacos. O ECG pode ser obtido em laboratório, com o uso de eletrodos, o \textit{gold-standard} do método, porém estudos \cite{Plews2017ComparisonMethods, Giles2016ValidityRest.} demonstram que as cintas disponíveis comercialmente com eletrodos embutidos, frequentemente utilizadas por atletas para medir seus treinamentos, têm precisão comparável à de um ECG em laboratório, viabilizando seu uso para projetos de pesquisa com um hardware fácil de ser obtido e usado. 
        
        \paragraph{} Outro método para obter a frequência cardíaca e derivar o HRV é através de fotopletismografia (PPG), amplamente utilizado em \textit{smartwatches} disponíveis comercialmente e até em \textit{softwares} que usam a câmera do celular posicionada sobre o dedo. Essa técnica mede a frequência cardíaca indiretamente, através da absorção de luz pela pele causada pela variação de fluxo sanguíneo. 
       
        \subsection {Regulação autonômica}
        
        \paragraph{} O nó sinoatrial, conhecido como o marcapasso natural do coração, gera aproximadamente 100 pulsos elétricos por minuto. No entanto, essa taxa não é regular, mas sim modulada por sinais oriundos do Sistema Nervoso Autônomo (SNA), o qual é composto por dois ramos opostos. O Sistema Nervoso Simpático (SNS) é responsável por reações a estímulos externos de estresse, desencadeando reações tais como o aumento da frequência cardíaca e do fluxo sanguíneo e a dilatação das pupilas. Ele é responsável por deixar o corpo pronto para reagir ao estímulo. O Sistema Nervoso Parassimpático, por sua vez, age como um freio a esse sistema quando não há estímulo externo, permitindo que o organismo relaxe, reduza a frequência cardíaca e retome as atividades normais, como a digestão ~\cite{Oweis2014QRSSurvey}. Ele também é responsável por manter o ritmo regular da respiração e seu principal componente, o nervo vago, tem papel crucial nos mecanismos de digestão, sono e regulação fisiológica. A ação concorrente desses dois sistemas cria as flutuações sutis nos intervalos detectadas pelos métodos de análise de HRV.
      
        \subsection {Usos na literatura}
        
            \paragraph{} A primeira descrição do HRV foi no contexto de um método não-invasivo de  deteção de angústia fetal ~\cite{Quintana2016GuidelinesCommunication}, contudo ele foi, desde então, adotado amplamente nas áreas de saúde e esporte. 
            
            \paragraph{} Sua importância é relevante na medicina esportiva, onde é usado como uma forma de melhorar a performance e monitorar a recuperação do atleta para reduzir as lesões causadas por excesso de treino. ~\cite{Oweis2014QRSSurvey, Shaffer2017AnNorms., Plews2017ComparisonMethods} Também é amplamente usado como marcador de estresse e condições de risco, assim como utilizado para técnicas de \textit{biofeedback} que tentam regular o estresse ~\cite{Vanitha2014HierarchicalVariability, Bernardi2000EffectsVariability, Prinsloo2011TheStress, Sasaki2014ConsciouslyActivity,  Quintana2016GuidelinesCommunication} e até mesmo como indicador de comorbidades psiquiátricas ~\cite{Quintana2016GuidelinesCommunication}. Pesquisas também sugerem que o HRV é importante para o funcionamento efetivo em ambientes complexos, impactando o funcionamento do córtex pré-frontal ~\cite{Hansen2004HeartDetraining, Luque-Casado2013CognitiveLevel}.
            
            \paragraph{} Estudos anteriores obtiveram sucesso aplicando classificadores sobre séries temporais de intervalos, empregando essa técnica para identificar anomalias no ritmo cardíaco ~\cite{Kampouraki2009HeartbeatMachines} e na detecção de sintomas de estresse ~\cite{Vanitha2014HierarchicalVariability, Sami2004ArtefactData}. O \textit{software Firstbeat} emprega o HRV para fornecer comercialmente um relatório identificando situações de estresse, relaxamento e exercício ao longo de 72h ~\cite{FirstbeatTechnologiesLtd.StressVariability}.
            
        \subsection {Métricas}
        
            \paragraph{} Os padrões mais amplamente utilizados para a análise de HRV foram definidos pela primeira vez por uma força-tarefa da European Society of Cardiology e da North American Society of Pacing Electrophysiology em 1996 ~\cite{TaskForceoftheEuropeanSocietyofCardiologytheNorthAmericanSocietyofPacing1996HeartUse}. A metodologia consiste em gravar uma janela de tempo de intervalos e calcular algumas métricas relacionadas com a ação do sistema nervoso autônomo sobre a frequência. 
            
            \paragraph{} As métricas se dividem no domínio do tempo, onde são calculadas estatísticas sobre a distribuição dos intervalos, como a média ou desvio-padrão e no domínio da frequência, onde são calculadas potências do espectro de densidade. Para o cálculo de métricas no domínio da frequência, é necessário regularizar a série temporal para aplicar o algoritmo de FFT. As principais métricas estão descritas nas Tabelas ~\ref{timedomain} e ~\ref{freqdomain}.  Cabe observar que a sigla de algumas métricas se refere a intervalos NN no lugar de RR para ressaltar que esses valores são normalizados.

            \begin{table}[h!]
                \centering
                \caption{Métricas calculadas sobre a  distribuição dos intervalos RR no domínio do tempo}
                \label{timedomain}
                \begin{tabular}{l | p{8cm}}
                MHR & média da frequência cardíaca (média móvel do inverso do valor dos intervalos) \\
                MRRI  & média dos valores dos intervalos                             \\
                SDNN  & desvio-padrão dos valores dos intervalos                     \\
                RMSSD & média RMS dos valores das diferenças entre dois intervalos consecutivos \\
                PNN50 & porcentagem dos intervalos consecutivos cuja diferença é
                superior a 50ms \\
                HR Max - HR Min & diferneça entre os valores extremos observados para a frequência cardíaca \\
                \end{tabular}
            \end{table}

            \begin{table}[h!]
                \centering
                \caption{Métricas calculadas sobre o espectro de potência da série de intervalos no domínio da frequência}
                \label{freqdomain}
                \begin{tabular}{l | p{8cm}}
                HF & valor absoluto, em $ms^2$, da potência do espectro na faixa 0.15–0.40Hz\\
                HFNU  & valor normalizado da potência do espectro na faixa 0.15–0.40Hz\\
                HFpeak & Frequência de pico do espectro na faixa 0.15–0.40Hz \\
                LF & valor absoluto, em $ms^2$, da potência do espectro na faixa 0.04–0.15Hz \\
                LFNU  & valor normalizado da potência do espectro na faixa 0.04–0.15Hz\\
                LFpeak & Frequência de pico do espectro na faixa 0.04–0.15Hz \\
                LF:HF & razão entre os valores de LF e HF \\
                VLF & valor absoluto, em $ms^2$, da potência do espectro na faixa 0.0033–0.04Hz \\
                ULF & valor absoluto, em $ms^2$, da potência do espectro na faixa $<= 0.003$Hz \\
                \end{tabular}
            \end{table}